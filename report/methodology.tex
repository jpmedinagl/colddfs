\section{Methodology}
\label{s:methodology}

\subsection{Development Environment}
The distributed file system simulator was implemented in C, chosen for its
low-level control over system resources. The development and testing were
conducted on a Linux-based system. The simulator uses standard POSIX API:
\begin{itemize}
    \item Process management (fork)
    \item UNIX domain sockets for inter-process communication
    \item File I/O operations for persistent storage
    \item Memory Management for buffer allocation.
\end{itemize}


\subsection{Experimental Setup}
All experiments were executed on a single physical machine. While this setup does
not capture network latency effects present in distributed systems, it allows for
controlled evaluation of algorithmic and architectural design decisions.

The simulation environment created separate processes for each data node,
with communication occuring through UNIX sockets to approximate the message-passing
behaviour of networked systems. Each node maintains its own directory within the host
file system, simulating independent isolated storage devices.

\subsection{Experimental Procedure}
For each experiment configuration:
\begin{enumerate}
    \item Initialize the simulator with specific parameters (data nodes, capacity,
    block size, allocation policy).
    \item Create workload based on operation type (ex. 50\% reads, 20\% writes, etc...)
    \item Collect timing data for each operations.
    \item Calculate aggregate performance metrics
    \item Clean up data node directories.
    \item Repeat for statistical validation.
\end{enumerate}

All experiments were designed to isolate the impact of individual parameters.